In this section, we will focus on modeling the airflow around the previously interpolated wing (both the upper and lower parts of the wing).
\subsection{Modelling the airflow}
The subject explains that the airflow is laminar. This means that it can be divided into two parts, and each air molecule moves along curves that do not intersect. Let $h_{min}^{m}$ and $h_{max}^{m}$ be the minimum and maximum height of the airflow, respectively. We assume that the airflow is disturbed by the wing in a vertical interval of $[[3h_{min}; 3h_{max}]]$, otherwise it is rectilinear. Finally, let $y$ be the curve modeling either the extrados or the intrados (by replacing $h_{max}$ with $h_{min}$ of the wing):
\begin{equation}
    \label{eq:equation_part3.1}
    y = f_{\lambda}(x) = (1 - \lambda)f(x) + \lambda \times h_{max} \times 3 \quad \forall \lambda \in [0,1] 
\end{equation}
Using the extrados $x/y$ $(ex, ey)$, the intrados $x/y$ $(ix, iy)$, the function that interpolates the points of the upper part of the wing (fint-supp), and the function that interpolates the points of the lower part of the wing (fint-inf), we obtain this figure \ref{fig:laminar}, which confirms the laminar nature of the airflow:
\begin{figure}[H]
  \centering
  \includegraphics[width=0.5\textwidth]{laminar_flow.png}
  \caption{Laminar flow above and below the wing}
  \label{fig:laminar}
\end{figure}
\subsection{Pressure map}
